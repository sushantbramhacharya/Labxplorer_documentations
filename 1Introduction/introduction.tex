\chapter{INTRODUCTION}
% (20% of Proposal Length)
\pagenumbering{arabic}

% Introduction: (20\% of Report Length)


\section{Introduction}
LabXplorer revolutionizes science education by providing an innovative virtual laboratory platform designed to transcend traditional learning methods. Tailored specifically for students and educators in STEM fields, LabXplorer aims to bridge gaps in practical science education by offering interactive simulations and experiments across diverse disciplines. This cutting-edge platform serves as a dedicated arena where learners can engage deeply with scientific concepts, conduct virtual experiments, visualize data, and collaborate seamlessly within their academic community.


\section{Problem Statement}


There are many general educational platforms available, but none are specifically designed to provide virtual laboratory experiences for science education. This means that students and educators often have to rely on traditional methods, which can be less effective for interactive learning and experimentation. Most general educational platforms do not have dedicated spaces for conducting virtual experiments. This makes it difficult for students to gain hands-on experience and apply theoretical knowledge in a practical setting. There are no specific tools available for creating and managing virtual lab simulations tailored to various scientific disciplines. This means that educators often have to use general-purpose simulation tools, which can be less effective for specific educational needs.

There is no specific platform that integrates various scientific disciplines into one comprehensive virtual laboratory environment. This means that students often have to use multiple platforms for different subjects, which can be cumbersome and disjointed. The lack of dedicated virtual lab platforms can make it difficult for students to engage deeply with the subject matter and for educators to track and assess practical skills development. These challenges can be even more pronounced for schools and institutions with limited access to physical laboratory resources, as they may not be able to provide adequate hands-on experiences for students.
\section{Objectives}
\begin{itemize}
    \item Create a virtual laboratory platform that enhances science education through interactive simulations and experiments across various disciplines.
\end{itemize}
\section{Scope}
% Scope and limitation

\begin{itemize}
    \item The platform should provide a virtual space for students and educators to conduct interactive simulations and experiments across various scientific disciplines.
    \item LabXplorer should facilitate collaborative learning through discussion forums enabling students to share insights and ask questions.
    \item The platform should be user-friendly and accessible, making it easy for students of all levels to engage in virtual laboratory activities.
    
\end{itemize}
\section{Report Organisation}
The material in this project report is organised into seven chapters. After this introductory chapter introduces the problem topic this research tries to address, chapter 2 contains the literature review of vital and relevant publications, pointing toward a notable research gap. Chapter 3 describes the methodology for the implementation of this project. Chapter 4 provides an overview of what has been accomplished. Chapter 5 contains some crucial discussions on the used model and methods. Chapter 6 mentions pathways for future research direction for the same problem or in the same domain. Chapter 7 concludes the project shortly, mentioning the accomplishment and comparing it with the main objectives.