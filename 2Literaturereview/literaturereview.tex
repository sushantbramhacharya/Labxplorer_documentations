\chapter{BACKGROUND AND LITERATURE REVIEW}

% (20\% of Report Length)

% a. Must be paraphrased without plagiarizing

% b. Must include the base papers\cite{Adhikari2020Dec}, and support the rationale of the project

% c. Must highlight the strengths and shortcomings of the works performed by other authors

\section{Background Study}

Traditional science education, reliant on textbooks, lectures, and physical labs, often struggles to provide comprehensive hands-on experiences due to limited resources, high costs, and safety concerns. These challenges are especially pronounced in under-resourced schools, where students lack opportunities to engage with scientific materials practically. Our design focuses on developing LabXplorer, an innovative platform designed to revolutionize science education by offering interactive simulations in chemistry, physics, electronics, and astronomy. By providing a virtual environment for conducting experiments, visualizing data, and collaborating within an academic community, LabXplorer aims to transcend traditional learning methods. This platform also features a profile system for showcasing user portfolios and resumes, emphasizing functionality and visual appeal. We are researching best practices in code collaboration, tools for code sharing, and messaging functions to create a robust, user-friendly platform that bridges gaps in practical science education and sets a new standard for STEM learning.
\section{Literature Review}
\subsection*{Teacher perception of Olabs pedagogy}
OLabs, as name says, offers a robust web-based platform encompassing simulations, animations, tutorials, and assessments, designed to enhance interactive and accessible learning experiences outside traditional laboratory settings. Emphasizing student-centered learning, inquiry-based approaches foster essential skills such as scientific thinking, evidence-based reasoning, and creative problem-solving, which are fundamental for knowledge creation and retention.\cite{chandrashekhar2020teacher}
\subsection*{How Khan Academy is changing the rules of education}
This paper briefly describes how can an online learning platfrom change the way our education system worka and improve on it.
\begin{itemize}
    \item Khan Academy offers free, online instructional videos covering various subjects, allowing students to learn at their own pace and revisit concepts.
    
    \item The platform uses analytics to provide real-time feedback, enhancing personalized learning experiences for both teachers and students.
    
    \item Khan Academy promotes a flipped classroom model where students watch videos at home and engage in problem-solving and discussions in class, fostering deeper understanding and collaboration.
    
    \item It democratizes education by providing high-quality instruction globally, irrespective of geographic location or socioeconomic status.
    
    \item The platform challenges traditional educational paradigms and suggests new possibilities for delivering effective education in the digital era.
\end{itemize}
Khan Acadamy being one of the main motivation for onile learning and educating.
\cite{thompson2011khan}
\subsection*{PhET: Interactive simulations for teaching and learning physics}
Perkins et al. (2006) introduce PhET, a collection of interactive simulations designed to enhance the teaching and learning of physics. These simulations aim to make abstract concepts more accessible and understandable through dynamic visualizations and interactive models. The authors emphasize the effectiveness of PhET in promoting conceptual understanding by allowing students to manipulate variables and observe real-time outcomes, thereby bridging the gap between theoretical concepts and practical application. They discuss the development process, which involves collaboration between physicists, educators, and software developers to ensure accuracy and educational efficacy. The article highlights PhET's versatility in catering to diverse learning styles and educational settings, promoting active learning and engagement. \cite{perkins2006phet}
\subsection*{An Introduction to HTML5 Game Development with Phaser.js}
It provides a comprehensive guide to creating 2D games using the Phaser.js framework. It covers setting up a development environment, understanding fundamental game concepts, and managing game states and assets. The book teaches how to implement physics and collision detection, create animations and visual effects, design user interfaces, and integrate audio. It emphasizes practical, project-based learning, guiding readers through real game development scenarios. Additionally, it offers debugging, optimization techniques, and deployment strategies for various platforms, making it an essential resource for both beginners and experienced developers looking to master HTML5 game development with Phaser.js.\cite{faas2017introduction}